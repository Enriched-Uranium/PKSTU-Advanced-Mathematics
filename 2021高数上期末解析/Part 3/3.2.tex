\paragraph{3.2} 
%设 $f(x)=\left\{\begin{aligned}&\sin x+2a\mathrm{e}^x,\quad \qquad \quad x<0\\&9\arctan x+2b(x-1)^3,x\geq 0\\\end{aligned}\right.$, 试确定 $a,b$ 的值, 使得函数 $f(x)$  在其定义域上可导.
%\paragraph{解析}
容易发现 $f(x)$ 的定义域为 $\mathbb{R}$, 且在 $x\neq 0$ 时 $f(x)$ 可导, 故只需考虑分段点处的情况.

若使 $f$ 在 $x=0$ 处可导, 则 $f$ 在 $x=0$ 处连续, 有 $\lim\limits_{x\to 0^{-}}f(x)=\lim\limits_{x\to0^{+}}f(x)=f(0)$. 又
\begin{equation}
\lim_{x\to 0^{-}}f(x)=\lim_{x\to 0^{-}}\sin x+2a\mathrm{e}^x=2a
\end{equation}
\begin{equation}
\lim_{x\to 0^{+}}f(x)=\lim_{x\to 0^{+}}9\arctan x+2b(x-1)^3=-2b
\end{equation}

由 (1), (2) 可得 $a=-b$.

因为 $f$ 在 $x=0$ 处可导, 有 $f'_{-}(0)=f'_{+}(0)$, 又
\begin{equation}
	f'_{-}(0)=\lim_{x\to 0^{-}}\dfrac{f(x)-f(0)}{x}=\lim_{x\to 0^{-}}\dfrac{\sin x+2a\mathrm{e}^x-2a}{x}=2a+1
\end{equation}
\begin{equation}
\begin{aligned}
f'_{+}(0)=\lim_{x\to0^{+}}\dfrac{f(x)-f(0)}{x}&=\lim_{x\to 0^{+}}\dfrac{9\arctan x+2b(x-1)^3+2b}{x}\\
&=\lim_{x\to 0^{+}}\dfrac{9\arctan x+2b\cdot x\cdot\left[(x-1)^2-(x-1)+1\right]}{x}\\
&=\lim_{x\to 0^{+}}9+2b\cdot(x^2-3x+3)=9+6b
\end{aligned}
\end{equation}

即 $2a+1=9+6b$, 解得 $a=1,b=-1$. 
