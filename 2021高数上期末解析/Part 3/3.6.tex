\paragraph*{3.6}  设被积函数为 $g(x)$, 则 $g(x)$ 有奇点 $x=0,x=2$. 设
\[
	I=\int_{-1}^3g(x)\,\mathrm dx=\int_{-1}^0g(x)\,\mathrm dx+\int_{0}^{2}g(x)\,\mathrm dx+\int_{2}^{3}g(x)\,\mathrm dx\overset{\mathrm{def}}{=}I_1+I_2+I_3
\] 
又 $f(0-0)=-\infty,f(0+0)=+\infty,f(2-0)=-\infty,f(2+0)=+\infty$, 故
\[
	I_1=\arctan f(x)\mid_{-1}^{0}=\arctan(f(0-0))-\arctan (f(-1))=-\dfrac{\pi}{2}-0=-\dfrac{\pi}{2}
\]
\[
	I_2=\arctan f(x)\mid_{0}^{2}=\arctan(f(2-0))-\arctan (f(0+0))=-\dfrac{\pi}{2}-\dfrac{\pi}{2}=-\pi
\]
\[
	I_3=\arctan f(x)\mid_{2}^{3}=\arctan(f(3))-\arctan(f(2+0))=\arctan\dfrac{32}{27}-\dfrac{\pi}{2}
\]

于是 $I=\arctan\dfrac{32}{27}-2\pi$.