\begin{enumerate}
    \item \(26\).\[\]

    注意到\[\boldsymbol{B} = [\boldsymbol{\alpha}_1, 2\boldsymbol{\alpha}_1 + 3 \boldsymbol{\alpha}_2 + 5\boldsymbol{\alpha}_3 , \boldsymbol{\alpha}_2 + 6 \boldsymbol{\alpha}_3] = [\boldsymbol{\alpha}_1, \boldsymbol{\alpha}_2 , \boldsymbol{\alpha_3}] \begin{pmatrix}
        1 & 2 &  \\ & 3 & 1 \\ & 5 & 6\\
    \end{pmatrix} = \boldsymbol{A}\begin{pmatrix}
        1 & 2 &  \\ & 3 & 1 \\ & 5 & 6\\
    \end{pmatrix}\]
    故\[|\boldsymbol{B}| = |\boldsymbol{A}| \left|\begin{pmatrix}
        1 & 2 &  \\ & 3 & 1 \\ & 5 & 6\\
    \end{pmatrix}\right| = 26\]

    \item \(-2\).\[\]
    
    由于实对称矩阵属于不同特征值的特征向量相互正交,故\[\langle \boldsymbol{\alpha}_1, \boldsymbol{\alpha}_2 \rangle = - 1+ 3 + x = 0\qquad \Longrightarrow\qquad x = -2\]

    \item \(\displaystyle \begin{pmatrix}
        c_{21} & c_{22} & c_{23} \\ 
        c_{11} & c_{12} & c_{13} \\ 
        c_{31} & c_{32} & c_{33} \\ 
    \end{pmatrix}   .
    \)\[\]
    
    \(\boldsymbol{A}^{15}\boldsymbol{C}\)是将\(\boldsymbol{C}\)的\(1,2\)行交换\(15\)遍,即交换\(\boldsymbol{C}\)的\(1, 2\)行。\(\boldsymbol{CB}^{16}\)是将\(\boldsymbol{C}\)的\(1,3\)列交换\(16\)遍,即不变。故总体的作用是将\(\boldsymbol{C}\)的\(1,2\)行交换。

    \item \((1, 0, -1)\).\[\]

    设\(\displaystyle \frac{x}{-1} = \frac{y - 1}{1} = \frac{z - 1}{2} = t\),即\[\left\{\begin{aligned}
        &x = -1 \\ &y = t + 1 \\ &z = 2t + 1
    \end{aligned}\right.\]
    带入平面方程有\[-2t + t + 1 - 2t - 4 = 0 \qquad \Longrightarrow\qquad t = -1 \qquad \Longrightarrow\qquad \left\{\begin{aligned}
        &x = 1 \\ &y = 0 \\ &z = -1
    \end{aligned}\right.\]

    \item \( t \in (0, 2).\)\[\]
    
    实二次型的矩阵为\[\boldsymbol{A} = \begin{pmatrix}
        t & & \\ & 1 & t \\ & t & 4\\
    \end{pmatrix}\]
    由正惯性指数为\(3\)可知\(\boldsymbol{A}\)是正定矩阵,故各阶顺序主子式大于\(0\),即\[\left\{\begin{aligned}
        &\Delta_1 = t > 0\\ &\Delta_2 = t > 0\\ & \Delta_3 = 4t - t^3 > 0
    \end{aligned}  \right. \qquad \Longrightarrow\qquad t \in (0, 2)\]




\end{enumerate}