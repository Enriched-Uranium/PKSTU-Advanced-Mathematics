\paragraph*{4.3}

(1). 由 $\lims_{x\to 0}\dfrac{f(x)}{x}=1$ 知 $f(0)=0$, 且 $f'(0)=1$. 故 $\exists a>0$, 使得 $f(a)>f(0)=0$.

同理, $f(1)=0,f'(1)=2$, $\exists b<1$, 使得 $f(b)<f(1)=0$, 且 $b\neq a$.

于是 $f(a)f(b)<0$, 由零点定理知 $\exists \xi \in (a,b)$, $f(\xi)=0$.

(2). 构造 $\mathscr F(x)=\e^{-x}f(x)$. 则 $\mathscr F(0)=\mathscr F(1)=\mathscr F(\xi)=0$. 由 Rolle 定理知: 存在 $\xi_1\in(0,\xi),\xi_2\in(\xi,1)$ 使得 $\mathscr F'(\xi_1)=\mathscr F'(\xi_2)=0$.

又 $\mathscr F'(x)=\e^{-x}[f'(x)-f(x)]$. 故 $\xi_1,\xi_2$ 是 $f'(x)-f(x)=0$ 的两个根.

构造 $\mathscr G(x)=\e^{x}\left[f'(x)-f(x)\right]$, 则 $\mathscr G(\xi_1)=\mathscr G(\xi_2)=0$. 故存在 $\eta\in(\xi_1,\xi_2)$ 使得 $\mathscr G'(\eta)=\e^{\eta}[f''(\eta)-f(\eta)]=0$. 即 $f''(\eta)-f(\eta)=0$, 原命题得证.