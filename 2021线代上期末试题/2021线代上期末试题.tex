\documentclass[11pt,a4paper]{ctexart}
\usepackage{tcolorbox,listings,hlist,enumerate,ulem,ifthen,ctex}
\usepackage[left=1cm,right=1cm,top=1.5cm,bottom=1.5cm]{geometry}
\usepackage{amscd,amssymb,amsfonts,amsbsy,amsmath,verbatim,color, mathrsfs,yhmath,tkz-euclide,chemfig,siunitx,circuitikz}
\usepackage[version=4]{mhchem}
\usepackage{asymptote}\usepackage{framed}
\usepackage{lastpage}
\usepackage{geometry} %调整页面边框
\geometry{a4paper,scale=0.8}%调整到80%
\usepackage{graphicx}
\usepackage{tikz}
\usepackage{multirow}%纵向合并表格
\usetikzlibrary{shapes.geometric,through,decorations.pathmorphing,arrows.meta,quotes,mindmap,shapes.symbols,shapes.arrows,automata,angles,3d,trees,shadows,automata,arrows,shapes.callouts,patterns,through,hobby}
\usetikzlibrary{intersections}%应用此 library 找到两条曲线的交点
\usetikzlibrary{calc} % 引入计算支持
\usepgflibrary{fpu}%调用 fpu 程序库,在计算线段与曲线、曲线与曲线的交点时使用这个库。
\newcommand{\RNum}[1]{\uppercase\expandafter{\romannumeral #1\relax}}%输入罗马数字
\usepackage{pgfplots}
\pgfplotsset{compat=1.17}
\usetikzlibrary{arrows}
\pagestyle{empty}
\usepackage{mathrsfs}
\usepackage{diagbox}%制作带斜线表头的宏包
\usepackage{pifont}%输入带圈数字
\usepackage{verbatim}%区间注释宏包
\usepackage{setspace}
\usepackage{fancyhdr}
\usepackage{tasks}
\settasks{label=\Alph*.,
	label-offset={0.5em},
	label-align=left,
	column-sep={2pt},
	item-indent={1.3em},before-skip={-0.7em},after-skip={-0.7em}}
\usepackage{ifthen}
\usepackage{array}

%---------- 选择题和填空题设置  --------------
  %选择题的4个选项,使用一个命令根据选项内容长度自动排版
\newlength{\lab}
\newlength{\lb}
\newlength{\lc}
\newlength{\ld}
\newlength{\lhalf}
\newlength{\lquarter}
\newlength{\lmax}
\newcommand{\xx}[4]{%%%%%%%%%
	\settowidth{\lab}{A.~#1~~~}
	\settowidth{\lb}{B.~#2~~~}
	\settowidth{\lc}{C.~#3~~~}
	\settowidth{\ld}{D.~#4~~~}
	\ifthenelse{\lengthtest{\lab > \lb}}  {\setlength{\lmax}{\lab}}  {\setlength{\lmax}{\lb}}
	\ifthenelse{\lengthtest{\lmax < \lc}}  {\setlength{\lmax}{\lc}}  {}
	\ifthenelse{\lengthtest{\lmax < \ld}}  {\setlength{\lmax}{\ld}}  {}
	\setlength{\lhalf}{0.5\linewidth}
	\setlength{\lquarter}{0.25\linewidth}
	\ifthenelse{\lengthtest{\lmax > \lhalf}}
	{%
		\begin{hlist}[pre skip=0pt,item skip=0pt,,item offset={1.5em}, label=\Alpha {hlisti}.,pre label={}]1
			\hitem #1
			\hitem #2
			\hitem #3
			\hitem #4
		\end{hlist}
	}  %%%
	{%%
		\ifthenelse{\lengthtest{\lmax > \lquarter}} % 
		{%
			\begin{hlist}[pre skip=0pt,item skip=0pt,item offset={1.5em}, label=\Alpha {hlisti}.,pre label={}]2
				\hitem #1
				\hitem #2
				\hitem #3
				\hitem #4
			\end{hlist}
		}
		{%
			\begin{hlist}[\parskip=0pt,pre skip=0pt,item skip=0pt,item offset={1.5em}, label=\Alpha {hlisti}.,pre label={}]4
				\hitem #1
				\hitem #2
				\hitem #3
				\hitem #4
			\end{hlist}
}}}

\newcommand{\tk}[2][2]{\uline{\makebox[#1cm][c]{%
				\ifanswer
				\textcolor{red}{#2}%
				\else
				\phantom{#2}%
				\fi}}}%填空答案
\newcommand{\kh}[1]{\hfill(\  {{%
			\ifanswer
			{\makebox[0.4cm][c]{\textcolor{red}{#1}}}%
			\else
			\phantom{\makebox[0.4cm][c]{#1} }%
			\fi}\  )}}%选择答案
	
\newif\ifanswer
\newcommand{\answer}[1]{\ifthenelse{\isodd{#1}}{\answertrue}{}}
\newcommand{\ty}[2]{\textcolor{blue}{\  {{%
				\ifty
				{%
				#1}%
				\else
				%\vspace*{8cm}%这里是空出的答题区域。也可以在exam-2022.tex中的题目下方插入合适的空白区域大小
				\fi}\  }}}%简答题答案/答案解析				
\newif\ifty
\newcommand{\tiyuan}[1]{\ifthenelse{\isodd{#1}}{\tytrue}{}}

\newcommand{\who}[1]{{{ %
			\ifwho
			{\hfill(\ \makebox[2.0cm][c]{\textcolor{blue}{#1}}\  )}%
			\else
			\phantom{\makebox[2.0cm][c]{#1} }%
			\fi}}}%(供题人:xxx)			
\newif\ifwho
\newcommand{\zuozhe}[1]{\ifthenelse{\isodd{#1}}{\whotrue}{}}
 \answer{1}%%1显示填空选择答案;0不显示
 \tiyuan{1}%%1显示简答题答案和解析文字;0不显示
 \zuozhe{1}%%1显示供题人;0不显示
\linespread{1.4}
%以下的这些包基本在setting文件里都有了,这里还是列出来不亏,请勿随意删除。请阅读一下settings文件,里面会发现很多在本说明中看不见的功能,篇幅限制无法一一注明。
\usepackage{ulem} %这个package是用来画删除线的
\usepackage{amssymb}
\usepackage{titlesec}%titlesec宏包调整section与正文间距
\usepackage{zref-user,zref-lastpage}%使用zref宏包,引用数字标签值和LastPage标签,感谢qingkuan大神指导
%\usepackage{times} %use the Times New Roman fonts
\usepackage{bigstrut}
\usepackage{enumerate}
\usepackage{amsmath,bm,amsthm,mathrsfs}
\everymath{\displaystyle}
\newcommand\dif{\mathop{}\!\mathrm{d}}
\def\d{\,\mathrm{d}}
\usepackage{tikz}%可以用于一些绘图
\usepackage{fancybox}
\usepackage{rotating}
\usepackage{tabularx}
\usepackage{wasysym}
\usepackage{color}

%%%%%%%%%%%%%%%%%%%%%%%%%%%%%%%%%%%%%%%
%增加新的一页,与自然形成的新一页相比,带一个题头。不需要题头就不用了
%%%%%%%%%%%
\def\d{\;\mathrm{d}}
\def\t{^\mathrm{T}}

\pagestyle{fancy} %页码显示
\renewcommand\headrulewidth{0pt}%隐藏页眉横线
\setlength\headheight{15pt}
  %  \cfoot{理科数学试卷\quad 第  \thepage 页(共 \pageref{LastPage}页)}
    \newcommand{\cndash}{\rule{0.2em}{0pt}\rule[0.35em]{1.6em}{0.05em}\rule{0.2em}{0pt}}%中文破折号
    \newcommand{\dotsbrack}{\dotfill (\qquad).}%选择题括号
   \newcommand{\blank}{\underline{\hspace{70pt}}}%填空题横线
 %%%%%%%%%%%%%%%%%%%%%%%%%%%%%%
 %%%%%%%%%%%%%%%%%%%%%%%%%%%%%%
\begin{document}
    \begin{center}
    	\begin{Large}
    		\begin{center}
    			\textbf{2021年线性代数与解析几何期末试题}\\
    			% \textbf{《高等数学》}\\
    			% {\normalsize 仲英学业辅导中心 · 监制}\\
% {\zihao{5} 

		  
% 		............................................装........................................订.......................................线............................................... \\[1mm]
% 	}


    		\end{center}
    	\end{Large}
    \end{center}\vspace{-5mm}




\begin{framed}



    \begin{large}	

    \begin{large}
			\noindent\textbf{一、填空题(共~5~题,每题~3~分)}
	\end{large}        
	\begin{enumerate}
		% \setcounter{enumi}{5}%从5开始计数,用于更改题号
        \item 设\(\boldsymbol{\alpha}_1, \boldsymbol{\alpha}_2, \boldsymbol{\alpha}_3\)为\(3\)元列向量。\(\boldsymbol{A} = [\boldsymbol{\alpha}_1, \boldsymbol{\alpha}_2 , \boldsymbol{\alpha}_3],\; \boldsymbol{B} = [\boldsymbol{\alpha}_1, 2\boldsymbol{\alpha}_1 + 3 \boldsymbol{\alpha}_2 + 5\boldsymbol{\alpha}_3 , \boldsymbol{\alpha}_2 + 6 \boldsymbol{\alpha}_3]\),且\(|\boldsymbol{A}| = 2\),则\(|\boldsymbol{B}| = \)\tk{}.

        \item 设\(\boldsymbol{\alpha}_1 = (-1, 1, 1)\t,\; \boldsymbol{\alpha}_2 =  (1, 3, x)\t\)是实对称矩阵\(\boldsymbol{A}\)的属于不同特征值所对应的特征向量,则\(x = \)\tk{}.
        
        \item 设矩阵\(\boldsymbol{A}\)由\(3\)阶单位矩阵\(\boldsymbol{E}\)交换\(1 , 2\)行得到,矩阵\(\boldsymbol{B}\)由单位矩阵\(\boldsymbol{E}\)交换第\(1, 3\)列得到,矩阵\(\boldsymbol{C} = \displaystyle \begin{bmatrix}
            c_{11} & c_{12} & c_{13} \\ 
            c_{21} & c_{22} & c_{23} \\ 
            c_{31} & c_{32} & c_{33} \\ 
        \end{bmatrix}\),则\(\boldsymbol{A}^{15} \boldsymbol{C} \boldsymbol{B}^{16} = \)\tk{}.

        \item 直线\(\displaystyle \frac{x}{-1} = \frac{y - 1}{1} = \frac{z - 1}{2}\)与平面\(2x + y - z - 3 = 0\)的交点是\tk{}.
        \item 设实二次型\(f(x_1, x_2, x_3) = tx_1^2 + x_2^2 + 2tx_2x_3 + 4x_3^2\)的正惯性指数为\(3\),则参数\(t\)的取值范围为\tk{}.

	\end{enumerate} 

    \vspace{4mm} \noindent \textbf{二、选择题(共~5~题,每题~3~分)}
	\end{large}
    \begin{enumerate}
        
        \item 设\(4\)元非齐次方程组\(\boldsymbol{AX} = \boldsymbol{\beta}\)的系数矩阵的秩为\(2\),\(\boldsymbol{X}_1,\; \boldsymbol{X}_2\)是\(\boldsymbol{AX} = \boldsymbol{\beta}\)的两个解,\(\boldsymbol{\alpha}_1,\; \boldsymbol{\alpha}_2\)是导出组\(\boldsymbol{AX} = \boldsymbol{0}\)的线性无关的解,则\(\boldsymbol{AX} = \boldsymbol{\beta}\)的通解为\kh{}
        
        \xx{$\displaystyle\frac{1}{2}(\boldsymbol{X}_1 - \boldsymbol{X}_2) + k_1 (\boldsymbol{\alpha}_1 + \boldsymbol{\alpha}_2) + k_2 \boldsymbol{\alpha}_2$}
        {$\displaystyle\frac{1}{2}(\boldsymbol{X}_1 + \boldsymbol{X}_2) + k_1 (\boldsymbol{\alpha}_1 + \boldsymbol{\alpha}_2) + k_2 \boldsymbol{\alpha}_2$}
        {$\displaystyle\boldsymbol{X}_1 + k_1(\boldsymbol{X}_1 - \boldsymbol{X}_2)  + k_2 \boldsymbol{\alpha}_2$}
        {$\displaystyle\boldsymbol{X}_1 + k_1 (\boldsymbol{X}_1 - \boldsymbol{X}_2  ) + k_2 \boldsymbol{\alpha}_2 + k_3 \boldsymbol{\alpha}_2$}

        \item 设矩阵\(\displaystyle\boldsymbol{A} = \begin{bmatrix}
            -2 & x & 0 \\ 2 & 0 & 2 \\ 3 & 1 & 1
        \end{bmatrix}\)与\(\displaystyle \boldsymbol{B}  = \begin{bmatrix}
            y & & \\ & 2 & \\ & & -2\\
        \end{bmatrix}\)相似,则参数\(x, y\)的值为\kh{} \xx{\(x = 0, y = -1\)}{\(x = 0, y = 1\)}{\(x = y = -1\)}{\(x = y = 0\)}

        \item 设\(\boldsymbol{A}, \boldsymbol{B}\)为同阶方阵,\(\boldsymbol{E}\)为单位矩阵,则下列说法正确的有多少个?\kh{}
        \begin{enumerate}
            \item 若\(\boldsymbol{A}^2 = \boldsymbol{O}\) ,则\((\boldsymbol{E} - \boldsymbol{A})^{-1} = \boldsymbol{E} + \boldsymbol{A}\)
            \item 若\(\boldsymbol{A}^2 = \boldsymbol{A}\) ,则\(\boldsymbol{A} = \boldsymbol{O}\)或\(\boldsymbol{A} = \boldsymbol{E}\)
            \item \(\boldsymbol{AX} = \boldsymbol{AY}\),且\(\boldsymbol{A}\)可逆,则\(\boldsymbol{X} = \boldsymbol{Y}\)
            \item \((\boldsymbol{A} + \boldsymbol{B})^2 = \boldsymbol{A}^2 + 2 \boldsymbol{AB} + \boldsymbol{B}^2\)
        \end{enumerate}\xx{1}{2}{3}{4}

        \item 设三个向量\(\boldsymbol{a}, \boldsymbol{b} , \boldsymbol{c}\)满足\(\boldsymbol{a} + \boldsymbol{b} + \boldsymbol{c} = \boldsymbol{0}\),那么\(\boldsymbol{a} \times \boldsymbol{b} = \)\kh{}
        \xx{\(\boldsymbol{b} \times \boldsymbol{a}\)}{\(\boldsymbol{c} \times \boldsymbol{b}\)}{\(\boldsymbol{b} \times \boldsymbol{c}\)}{\(\boldsymbol{a} \times \boldsymbol{c}\)}

        \item 设向量组\(\boldsymbol{\alpha}_1, \boldsymbol{\alpha}_2 , \boldsymbol{\alpha}_3\)线性无关,\(n\)元向量\(\boldsymbol{\beta}_1\)可以由\(\boldsymbol{\alpha}_1, \boldsymbol{\alpha}_2 ,\boldsymbol{\alpha}_3\)线性表示,\(n\)元向量\(\boldsymbol{\beta}_2\)不能由\(\boldsymbol{\alpha}_1, \boldsymbol{\alpha}_2 ,\boldsymbol{\alpha}_3\)线性表示,则下列说法正确的是\kh{}
        \xx{\(\boldsymbol{\alpha}_2, \boldsymbol{\alpha}_3, \boldsymbol{\beta}_1\)线性无关}{\(\boldsymbol{\alpha}_2, \boldsymbol{\alpha}_3, \boldsymbol{\beta}_2\)线性无关}{\(\boldsymbol{\alpha}_2, \boldsymbol{\alpha}_3, \boldsymbol{\beta}_1 + \boldsymbol{\beta}_2\)线性相关}{\(\boldsymbol{\alpha}_2, \boldsymbol{\alpha}_3, \boldsymbol{\beta}_1, \boldsymbol{\beta}_2\)线性相关}

	\end{enumerate}

	   


	\begin{large}
		\noindent\textbf{三、(8分)}
	\end{large}

    设线性方程组\[\left\{\begin{aligned}
        x_1&  -&2x_2 &+ & x_3& +& x_4&- & 5  x_5 & = & 3\\
        2x_1&  -&4x_2 &+ &3 x_3&+ & x_4&- &11  x_5 & = & 6\\
        x_1&  -&2x_2 &+ &2 x_3&  +&3x_4& -&12 x_5 & = & 6\\
        3x_1&  -&6x_2 &+ &3 x_3&  +&2x_4&  -&13x_5 & = & t\\
    \end{aligned}\right.\]有解,求参数\(t\)以及方程组的结构式通解。\[\] \[\]\[\]\[\]\[\]\[\]\[\]\[\]\[\]\[\]

    \begin{large}
        \noindent\textbf{四、(8分)}
    \end{large}
    设实矩阵\(\boldsymbol{A} = \displaystyle \begin{bmatrix}
        1 & & & \\ & 1 & & \\ 1 &  & 1 & \\ & -3 & & 4
    \end{bmatrix}\),且\(\boldsymbol{AXA}^* = 8\boldsymbol{XA}^{-1} + 12 \boldsymbol{E}_4\),求矩阵\(\boldsymbol{X}\).\[\] \[\]\[\]\[\]\[\]\[\]\[\]\[\]\[\]\[\]

    \begin{large}
        \noindent\textbf{五、(9分)}
    \end{large}
    设\(\displaystyle\boldsymbol{A} = \begin{bmatrix}
        -1 & 1 & & \\ 1 & -1 & & \\ & & -1 & 1 \\ & & & -1\\
    \end{bmatrix}\)求\(\boldsymbol{A}^{2014}\)以及\(|\boldsymbol{A}^{2014}|\).\[\] \[\]\[\]\[\]\[\]\[\]\[\]\[\]\[\]\[\]

    \begin{large}
        \noindent\textbf{六、(9分)}
    \end{large}
    设向量组\(\boldsymbol{\alpha}_1 = (1, 2, 1)\t ,\; \boldsymbol{\alpha}_2 = (1, 3, 2) \t ,\; \boldsymbol{\alpha}_3 = (1, a, 3) \t \)为\(\mathbb{R}^3\)的一个基,\(\boldsymbol{\beta} = (1, 1, 1)\t \)在这个基下的坐标为\((b, c, 1)\t\).\begin{enumerate}
        \item 求\(a, b, c\);
        \item 证明\(\boldsymbol{\alpha}_2, \boldsymbol{\alpha}_3 , \boldsymbol{\beta}\)为\(\mathbb{R}^3\)的一个基,并求\(\boldsymbol{\alpha}_2, \boldsymbol{\alpha}_3, \boldsymbol{\beta}\)到\(\boldsymbol{\alpha}_1, \boldsymbol{\alpha}_2 , \boldsymbol{\alpha}_3\)的一个过渡矩阵。
    \end{enumerate}\[\] \[\]\[\]\[\]\[\]\[\]\[\]\[\]\[\]\[\]

    \begin{large}
        \noindent\textbf{七、(9分)}
    \end{large}
    在\(\mathbb{R}^3\)中,对于任意向量\(\boldsymbol{\alpha} = (x, y, z)\t\),规定\(T(\boldsymbol{\alpha}) = (x - y, y - z, z) \t.\)\begin{enumerate}
        \item 求线性变换\(T\)在基\(\boldsymbol{\alpha}_1 = (0, 0, 1)\t, \; \boldsymbol{\alpha}_2 = (0, 1, 1)\t,\; \boldsymbol{\alpha}_3 = (1, 1, 1)\t\)下的矩阵.
        \item 求\(T\)的值域和秩.
    \end{enumerate}\[\] \[\]\[\]\[\]\[\]\[\]\[\]\[\]\[\]

    \begin{large}
        \noindent\textbf{八、(10分)}
    \end{large}
    已知点\(P(2, 0, 1)\)和直线\(L : \left\{\begin{aligned}
        x - y - 4z + 12 = 0 \\ 2x + y - 2z + 3 = 0
    \end{aligned}\right.\)
    \begin{enumerate}
        \item 将直线\(L\)化为对称式方程;
        \item 求点\(P\)关于直线\(L\)的对称点.
    \end{enumerate}\[\] \[\]\[\]\[\]\[\]\[\]\[\]\[\]\[\]

    \begin{large}
        \noindent\textbf{九、(13分)}
    \end{large}
    \begin{enumerate}
        \item 用正交线性变换化实二次型\[f(x_1, x_2, x_3) = 3x_1^2 + 3x_2^2 - 2x_1x_2 + 6x_1x_3 - 6 x_2x_3\]为标准型,并写出所用的正交变换;
        \item 求二次型\(f(x_1, x_2, x_3)\)的规范型.
    \end{enumerate}\[\] \[\]\[\]\[\]\[\]\[\]\[\]\[\]\[\]

    \begin{large}
        \noindent\textbf{十、(4分)}
    \end{large}设\(n\)阶方阵\(\boldsymbol{A} = \boldsymbol{E}_n - \boldsymbol{\alpha\alpha} \t\),其中\(\boldsymbol{\alpha}\)是\(n\)元非零列向量,\(\boldsymbol{E}_n\)为\(n\)阶单位矩阵.证明\begin{enumerate}
        \item \(\boldsymbol{A}^2 = \boldsymbol{A}\)的充要条件是\(\boldsymbol{\alpha}\t \boldsymbol{\alpha} = 1\)
        \item 当\(\boldsymbol{\alpha}\t \boldsymbol{\alpha} = 1\)时,矩阵\(A\)  为降秩矩阵。
    \end{enumerate}\[\] \[\]\[\]\[\]\[\]\[\]\[\]\[\]\[\]


    \end{framed}
  
\end{document}